\documentclass{article}

\usepackage{style_template}

\begin{document}
\title{Аналитический отчет по ...} % заголовок отчета
\author{\itshape Иванов И.И.} % имя автора работы
\date{} % просим LaTeX не указывать дату, так как будет
        % использован наш вариант оформления даты, описанный в стилевом файле
\maketitle % создать заголовок

\thispagestyle{fancy} % задает стиль страницы

\tableofcontents

\section{Расчет усталостной долговечности по модели J.W. Miles}

Модель Miles \cite{miles-1954}, строго говоря, применима только к \emph{узкополосным} случайным процессам -- то есть к процессам с узким энергетическим спектром -- поэтому при нагружении \emph{широкополосными} процессами -- соответственно процессами с относительно широким энергетическим спектром -- модель дает чрезмерно консервативные оценки усталостной долговечности
\begin{align}\label{eq:miles}
	Y_{NB}(s_{\sigma}) = \Big[ \, \dfrac{f}{C} \, ( \sqrt{2} s_{\sigma} )^m \, \Gamma \bigg( 1 + \dfrac{m}{2} \bigg) \, \Big]^{-1}, \ C = \sigma_{-1\text{д}}^m N_0,
\end{align}
где $ f $ -- эффективная частота случайного процесса, Гц; $ \sigma_{-1\text{д}} $ -- предел выносливости детали, МПа; $ s_{\sigma} $ -- стандартное отклонение случайного процесса, МПа; $ m $ -- тангенс угла наклона левой ветви кривой выносливости; $ \Gamma(\cdot) $ -- гамма-функция.

Усталостная долговечность по модели \eqref{eq:miles} составляет $ Y_{NB} = $...

\begin{thebibliography}{99}\addcontentsline{toc}{section}{Список литературы}
	\bibitem{miles-1954}{\emph{Miles J.W.} On structural fatigue under random loading // Journal Aueronaut Science. 1954. V. 21. P. 753 -- 762.}
\end{thebibliography}

\end{document}