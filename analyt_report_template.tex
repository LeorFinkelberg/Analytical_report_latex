\documentclass{article}

\usepackage{basic_template}

\begin{document}

\title{Аналитический отчет по ...} % заголовок отчета
\author{\itshape Иванов И.И.} % имя автора работы
\date{} % просим LaTeX не указывать дату, так как будет
        % использован наш вариант оформления даты, описанный в стилевом файле
\maketitle % создать заголовок

\thispagestyle{fancy} % задает стиль страницы

В этой части можно разместить аннотацию к отчету. \TeX -- это издательская система компьютерной верстки, предназначенная для набора ...

% \shorttableofcontents{Краткое содержание}{1}

\tableofcontents

\section{Пример многострочной формулы}
    Для набора сложных многострочных формул используются различные окружения, например, окружение \texttt{multline}
	\begin{multline}\label{eq:FunRasp}
		F_{\zeta}(z)=P[\,\zeta\leqslant z\,] = \int\!\!\!\int_{x/y\leqslant z}f_X(x;n)f_Y(y;m)\,dxdy =\\ \dfrac{1}{2^{(n+m)/2}\Gamma(n/2)\Gamma(m/2)}\int\!\!\!\int_{x/y\leqslant z}x^{n/2-1}y^{m/2-1}\exp\left( -\frac{x}{2} \right) \exp\left( -\frac{y}{2} \right) \,\mathrm{d}x \, \mathrm{d}y.
	\end{multline}

\section{Пример группового размещения формул}

Несколько формул можно разместить в одной группе с помощью окружения \texttt{gather}
\begin{gather}
	\sum_{j \in \mathbf{N}} b_{ij} \hat{y}_{j} = \sum_{j \in \mathbf{N}} b_{ij}^\lambda \hat{y}_j + (b_{ii} - \lambda_i)\hat{y}_i \hat{y},\notag \\
	\det \mathbf{K}(t=1, t_1, \ldots, t_n) = \sum_{I \in \mathbf{n} } (-1)^{|I|} \prod_{i \in I} t_i \prod_{j \in I} (D_j + \lambda_j t_j) \det \mathbf{A}^{(\lambda)} (\, \overline{I} | \overline{I} \,) = 0,\tag{$a$} \\
	\mathbb{F} = \sum_{i=1}^{\left[ \frac{n}{2}\right] } \binom{ x_{i,i+1}^{i^2}}{ \left[ \frac{i+3}{3} \right]} {{\sqrt{\mu(i)^\frac{3}{2} (i^2-1)}} \over\displaystyle {\sqrt[3]{\rho(i)-2} + \sqrt[3]{\rho(i)-1}} }, \tag{$b$}
\end{gather}

\section{Простая однострочная формула}

Теорема Хинчина-Винера утверждает, что спектральная плотность мощности стационарного в широком смысле случайного процесса представляет собой преобразование Фурье от соответствующей автокорреляционной функции
\begin{equation*}
    S_{xx}(f) = \int\limits_{-\infty}^{\infty}  \, r_{xx} (\tau) e^{-j 2 \pi f \tau} \mathrm{d} \tau,\ \text{где}\ r_{xx}(\tau) = \mathbb{E}[\,x(t) \, x^{*}(t - \tau)\,].
\end{equation*}

\end{document}
